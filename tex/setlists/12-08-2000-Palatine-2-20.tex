\begin{supertabular}{p{0.03\textwidth}p{0.92\textwidth}}
 \textbf{S1}:  &                                                                                                                                                                                                           Intro\textsuperscript{} \textgreater \enspace Front Porch\textsuperscript{} \textgreater \enspace A Go Go\textsuperscript{} \textgreater \enspace Out Of Order\textsuperscript{}, \enspace Muff II: The Revenge\textsuperscript{} \textgreater \enspace Q*Bert\textsuperscript{} \textgreater \enspace FF[1]\textsuperscript{} \textgreater \enspace All Things Ninja\textsuperscript{}  \enspace  \\
 \textbf{S2}:  &  All In Time[2]\textsuperscript{} \textgreater \enspace Sociable Jimmy\textsuperscript{}, \enspace Africa\textsuperscript{}, \enspace Example 1[3]\textsuperscript{} \textgreater \enspace Flying\textsuperscript{} \textgreater \enspace Example 1\textsuperscript{} \textgreater \enspace August\textsuperscript{} \textgreater \enspace Young Lust[4]\textsuperscript{} \textgreater \enspace Nachos for Two\textsuperscript{} \textgreater \enspace Walking On The Moon\textsuperscript{} \textgreater \enspace Front Porch\textsuperscript{} \textgreater \enspace Slacker\textsuperscript{}  \enspace  \\
\end{supertabular}
