\begin{supertabular}{p{0.03\textwidth}p{0.92\textwidth}}
 \textbf{S1}:  &  Andy's Last Beer\textsuperscript{}, \enspace FF\textsuperscript{} \textgreater \enspace Kimble\textsuperscript{}, \enspace Syncopated Strangers[1]\textsuperscript{}  \enspace  \\
\end{supertabular}
